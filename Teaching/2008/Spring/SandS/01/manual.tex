\documentclass[11pt,a4paper,article,oneside]{memoir}
\usepackage[danish,english]{babel}
\usepackage[ansinew]{inputenc}
\usepackage{url}
\usepackage[bookmarks=true,colorlinks=true,pagebackref]{hyperref}
\usepackage{memhfixc}

\title{Dat2/F6S/SW4: Syntaks og semantik\\ Manual}
\author{Uli Fahrenberg}
\date{Foråret 2008}

\pagestyle{plain}

\begin{document}

\begin{titlingpage}

\maketitle

\tableofcontents
\end{titlingpage}

\chapter{Om manualen}
 
Dette er manualen for kurset `Syntaks og semantik.' På de følgende sider
vil jeg prøve at besvare alle de vigtige og typiske spørgsmål, du måtte
have som studerende.  \textbf{Læs denne manual grundigt igennem ved
  kursets start og læs i den igen, hvis der er noget du kommer i tvivl
  om.}

Manualen følger kurset, så denne manual er en opdateret udgave af
manualen fra foråret 2007, som i sig selv var en opdateret udgave af
Hans Hüttels manual fra foråret 2006. Forfatteren takker Hans for det
store arbejde han har lagt i manualen og i kurset generelt.
    
\chapter{Om kursets indhold}

\section{Hvilke emner rummer kurset?}

I dette kursus stifter vi bekendtskab med grundlæggende matematiske
modeller for programmeringssprogs syntaks og semantik. Teorien om
programmeringssprog er et centralt område af datalogi. Det er vigtigt, når
man skal designe et nyt programmeringssprog, lære et nyt
programmeringssprog, skrive en compiler eller fortolker eller beskæftige
sig med verifikation af programmer.

Kurset falder i to hoveddele, med ca.\ 7 minimoduler til hver:
\begin{itemize}
\item \emph{Syntaksdelen} af kurset handler om hvordan man kan beskrive et
  programmeringssprogs \emph{form} -- hvornår er en programtekst `fejlfri'
  og hvordan kan man finde ud af om en programtekst er `fejlfri'?
\item \emph{Semantikdelen} handler om programmeringssprogs \emph{adfærd}
  -- hvordan kan man beskrive udførelsen af et program uden at skulle tale
  om en konkret implementation med alle dens krummerlyrer og
  maskinafhængigheder?
\end{itemize}
Den opmærksomme læser vil have bemærket at $15- (7+ 7)= 1$, dvs. der
er ca.\ én overskydende minimodul. Denne sidste kursusgang skal vi
bruge på at opnå en dybere teoretiske forståelse for et af de helt
centrale emner i kurset, nemlig \emph{rekursion og fikspunkter}.

\section{Hvad kan det bruges til?}

Kursets indhold kan umiddelbart anvendes til f.eks.\ beskrivelse og design
af programmeringssprog, se næste afsnit. Det lægger også grobund for
kurset `Kompleksitet og beregnelighed' på næste semester, der beskæftiger
sig med hvad det egentlig er computere kan beregne (og hvad de \emph{ikke}
kan).

På et andet plan vil den \emph{matematiske modenhed} I (forhåbentlig)
opnår gennem dette kursus være jer til gavn under hele jeres resterende
uddannelse.

\section{Hvordan hænger kursets indhold sammen med kurset `Sprog og
  oversættere'?}

Programmeringssprogs form og adfærd er også en del af kurset `Sprog og
oversættere' som undervises af Bent Thomsen. Men hvor vores kursus
fokuserer på teoretiske aspekter ved \emph{beskrivelse} af
programmeringssprog, handler Bents kursus om \emph{design} af
programmeringssprog og konstruktion af oversættere. Vores kursus udgør
således et vigtigt teoretisk fundament for Bents kursus, og omvendt kan
Bents kursus give konkrete eksempler på ting fra vores kursus.

Desuden er de abstraktioner vi vil snakke om nyttige i en del sammenhænge
udenfor oversætterkonstruktion, f.eks. i områder som programverifikation
eller genkendelse af naturlige sprog.

\section{Hvad er målene for kurset?}

Et vigtigt mål med kurset er selvfølgelig at I skal forstå og kunne
anvende de begreber og værktøjer der bliver introduceret. Men det er ikke
det eneste: Et ligeså vigtigt formål er at I trænes i \emph{matematisk
  modenhed}.

Matematisk modenhed er et vidt begreb som er svært at definere. I forhold
til vores kursus drejer det sig mest om det der kaldes henholdsvist
\emph{symbolkompetence} og \emph{ræsonnementskompetence}: I trænes i at
læse og skrive matematisk symbolsprog og i at relatere det til naturligt
sprog, og i at læse og skrive matematiske argumenter, specielt
beviser. Begge kompetencer er vigtige indenfor datalogi, specielt når man
beskæftiger sig med mere grundlæggende aspekter af datalogiske
problemstillinger.

Mere præcist er målene for kurset at I
\begin{itemize}
\firmlist
\item lærer at formidle jer mundtligt inden for kursets emner,
\item lærer at formidle jer skriftligt indenfor kursets emner, herunder at
  I bliver fortrolige med anvendelsen af matematisk notation,
\item lærer at formidle jer præcist,
\item får et \emph{aktivt} forhold til kursets emner,
\item får overblik over kursets emner og hvordan de hænger sammen,
\item opnår et præcist kendskab til alle centrale begreber i kurset,
\item opnår et præcist kendskab til områdets centrale resultater, herunder
  hvordan man beviser dem,
\item lærer at relatere de centrale begreber og resultater til hinanden,
  og
\item får kompetence i brug af matematisk modellering som redskab i
  sprogdesign og sprogimplementation.
\end{itemize} 


\chapter{Om kursets form}

\section{Hvordan er en kursusgang opbygget?} 

En kursusgang i `Syntaks og semantik' kommer til at forløbe således:
\begin{description}\item[8:15--10:00] Forelæsning
\item[10.10--12:00] Opgaveregning
\end{description} 

\section{Hvad forberedelse forventes der til kursusgangene?}

Jeg forventer at I hurtigt skimter forelæsningens stof \emph{inden}
forelæsningen, og læser det grundigt igennem efter kursusgangen. Jeg
vil en gang imellem bruge noget af forelæsningstiden til
«spørgetime,» hvor I forventes at stille spørgsmål til bøgernes
eller min fremstilling af stoffet.

\section{Hvordan forløber opgaveregningen?} 
 
\emph{Opgaveregningen er den vigtigste del af undervisningen.} Regn
opgaverne i fællesskab; det har vist sig at være helt klart den bedste
måde. Bestem evt.\ en ord- og kridtfører.

\emph{Grupperummene er undervisningslokaler.} Studerende, der \emph{ikke}
deltager i opgaveregningen, kan derfor ikke opholde sig i grupperummene
under opgaveregningen.

\section{Er der afleveringsopgaver?}

Jeg vil stille to større opgaver i løbet af kurset, som I forventes at
bruge en del af hver kursusgangs opgaveregningstid på. Disse skal
afleveres \emph{gruppevist} og vil hver danne grundlag for et af
eksamensspørgsmålene.
 
De studerende der følger kurset som PE-kursus kan vælge at erstatte en
eller begge opgave(r) med en der relaterer sig direkte til deres projekt.
 
\chapter{Kursusmateriale} 
 
\section{Hvilken bog skal jeg bruge?} 
 
I \emph{syntaksdelen} benytter vi os af følgende bog:
\begin{quote}
  Michael Sipser: \emph{Introduction to the Theory of Computation}, Second
  Edition, PWS Publishing Co. 2005.
\end{quote}
Der findes en ældre udgave af bogen. Lad være med at købe den -- den er
forældet. Alt materiale i dette kursus henviser til den nye udgave.
 
\medskip\noindent I \emph{semantikdelen} bruger vi en note af min kollega
Hans Hüttel:
\begin{quote}
  Hans Hüttel: \emph{Pilen ved træets rod},
  Aalborg Universitet 2007.
\end{quote}
Den vil kunne købes i boghandelen i løbet af februar. (Vi skal jo ikke
bruge den før ottende kursusgang.)

\section{Hvad er kursets hjemmeside?}

Den er her:
\begin{quote}
  \url{http://sands08.twoday.net}
\end{quote}

\noindent Jeg har for vane at bruge blogs som kursushjemmesider, da det
\begin{enumerate}
\item gør livet nemmere for mig og
\item giver \emph{jer} mulighed for at \emph{kommentere} kurset. Gør
  endelig brug af den mulighed; det er en god måde at feedbacke (findes
  det ord?) på.
\end{enumerate}

\section{Er der en kursusplan?}
Sidste år så den ud som nedenfor, og jeg regner ikke med at vi laver meget
om:

\begin{center}
  \begin{tabular}{rp{7cm}@{\hspace{1em}}l}
    \multicolumn{3}{l}{\textbf{Syntax -- regulære sprog}} \\\toprule
    1 & Introduktion; sprog; regulære udtryk & Sipser afs. 0.2, 1.3
    \\ \hline
    2 & Endelige automater & Sipser afs. 1.1 \\ \hline
    3 & Nondeterministiske automater & Sipser afs. 1.2 \\ \hline 
    4 & Sprog der \emph{ikke} er regulære & Sipser afs. 1.4  \\
    \multicolumn{3}{l}{\textbf{\rule{0pt}{3ex}Syntax -- kontekstfrie sprog}} \\\toprule
    5 & Kontekstfrie grammatikker & Sipser afs. 2.1 \\ \hline
    6 & Pushdown-automater & Sipser afs. 2.2 \\ \hline
    7 & Sprog der \emph{ikke} er kontekstfrie & Sipser afs. 2.3 \\
    \multicolumn{3}{l}{\textbf{\rule{0pt}{3ex}Semantik}} \\\toprule
    8  & Introduktion til (operationel) semantik & HH, kap. 1 og 3 \\ \hline
    9  & Operationelle semantikker for \textbf{ Bims} & HH, kap. 4 \\ \hline 
    10 & Udvidelser af \textbf{ Bims} & HH, kap. 5 \\ \hline
    11 & Blokke og procedurer & HH, kap. 6 \\ \hline 
    12 & Parametermekanismer & HH, kap. 7 \\ \hline
    13 & Semantikopgaven & \\
    \multicolumn{3}{l}{\textbf{\rule{0pt}{3ex}Teoretisk grundlag}} \\\toprule
    14 & Rekursive definitioner & HH, kap. 14 \\\hline
    15 & Denotationel semantik & HH, kap. 13
  \end{tabular}
\end{center}

\chapter{Om eksamen}
 
\section{Hvordan bliver eksamen?}
  
Eksamen er mundtlig og varer 20 minutter per studerende. Der er et antal
eksamensspørgsmål som I kender på forhånd. Dem trækker I et af, hvorefter
I får 20 minutter til at forberede en besvarelse af spørgsmålet der
forventes at vare 10 til 15 minutter. Eksamenspræstationen bedømmes af en
\emph{ekstern} censor, og der gives karakter.

Eksamenspensum og -spørgsmålene vil blive fastlagt ved tredjesidste
kursusgang.  Syntaks- og semantikopgaverne er del af eksamenspensum.
 
\section{Hvad skal jeg kunne til eksamen?} 
 
Når censor og jeg bedømmer en eksamenspræstation, forsøger vi at besvare
en række spørgsmål:
\begin{description}
\item[\quad Har den studerende  overblik?]
  
  Her ser vi efter i hvor høj grad I kan gøre rede for
  \begin{itemize}
  \item hvilke begreber der er væsentlige.
  \item hvorfor det netop er disse begreber, der er de væsentlige.
  \item hvilke resultater der er væsentlige.  I et kursus som \emph{Syntaks
      og semantik} er de vigtige resultater normalt \emph{sætninger}.
  \item hvorfor det netop er disse resultater, der er de væsentlige.
  \item hvilke sammenhænge der er mellem denne del af stoffet og andre dele
    af stoffet.
  \end{itemize}
  
\item[\quad Kan den studerende formulere sig præcist?]
  
  Her ser vi efter i hvor høj grad I kan gøre rede for
  \begin{itemize}
  \item hvordan de vigtige begreber er defineret. I et kursus som
    \emph{Syntaks og semantik} bliver de vigtige begreber indført ved
    præcise \emph{definitioner}.
  \item hvad de forskellige størrelser, der indgår i en definition,
    betegner.
  \item hvor og hvordan vi bruger definitionerne.
  \item hvorfor definitionerne er formuleret som de er.
  \item hvordan de vigtige sætninger er formuleret, og hvilke antagelser der
    er gjort i sætningernes formulering.
  \end{itemize}
  
\item[\quad Kan den studerende tænke om emnet og forstå dets tankegange?]
  
  Her ser vi efter i hvor høj grad I kan gøre rede for
  \begin{itemize}
  \item strukturen i et ræsonnement, hvad de væsentlige ideer er og hvorfor
    de er væsentlige. I et kursus som \emph{Syntaks og semantik} er der især
    vigtige ræsonnementer i \emph{beviser} for sætninger.
  \item hvorfor vi får bevist vores sætning ved at foretage netop disse
    bevisskridt.
  \item hvor antagelser og definitioner bliver brugt i et ræsonnement.
  \item hvilke bevisteknikker der er vigtige i det aktuelle emne.
  \item hvilke konsekvenser en sætning har.
  \end{itemize}
  
\item[\quad Er den studerende selvstændig?]
  
  Her ser vi efter i hvor høj grad I kan formulere jer selvstændige. Vi
  ser efter
  \begin{itemize}
  \item i hvor høj grad I kan tale om emnet uden at læse højt eller skrive
    af fra dispositionen.
  \item hvad det er, I bruger dispositionen til. (Er det til at hjælpe med
    struktur? Er det for at få styr på detaljer? Skal alle de vigtige
    begreber læses op/skrives af?)
  \item hvordan I reagerer, hvis I bliver præsenteret for en
    problemstilling (f.eks. et eksempel), der ikke er kendt på forhånd.
  \end{itemize}
\end{description}

\section{FAQ}

\subsection{Hvorfor skal vi lære om alle de små opfundne legetøjssprog?
  Hvorfor kan vi ikke se på hele Java eller C eller \dots?}

Semantikdelen af kurset handler om hvordan man beskriver forskellige
fænomener set isoleret, og derfor er det godt at vælge simple sprog der
kun rummer disse fænomener. Senere skal I lære at betragte fænomenernes
samspil i stadigt mere komplekse sammenhænge. En analogi: I et indledende
kursus i mekanisk fysik lægger man heller ikke ud med at beskrive
komplicerede mekaniske systemer med gnidningsmodstand, tyngdekraft osv.
inddraget på samme tid.

\subsection{Hvorfor er der alle de beviser?}

I et universitetsstudium skal man bl.a.\ erhverve sig en teoretisk indsigt
i studiets områder, og dette indebærer at man skal kende til
teoriområdernes argumentation.  Angrebsvinklen i dette kursus er primært
teoretisk, og de teorier, vi skal se på, er matematisk funderede teorier.
Resultaterne i matematisk orienterede discipliner er matematiske
\emph{sætninger}, og argumentationsformen er matematiske \emph{beviser}.
Så derfor er det denne argumentationsform, du vil få at se.  Andre
discipliner i datalogi benytter sig af andre argumentationsformer.
 
Samtidig er der også en læringsbetinget grund til at gøre beviserne for
kursets sætninger til genstand for en mere nøje betragtning. Matematiske
sætninger siger nemlig typisk noget om, \emph{hvordan begreber hænger
  sammen}, og man kan derfor lære at relatere kursets begreber til
hinanden ved at betragte beviserne. Et eksempel er resultatet om
sammenhængen mellem pushdown-automater og kontekstfrie grammatikker
(Sipser Lemma 2.15): Beviset fortæller præcis hvordan man kan konstruere
en grammatik ud fra en automat, og ved at I sætter jer ind i beviset, får
I forhåbentlig en bedre forståelse af både grammatikker og automater.

\end{document}
