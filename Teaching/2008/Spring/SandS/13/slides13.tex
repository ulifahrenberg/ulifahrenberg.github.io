\documentclass{beamer}

\newcommand*\lektion{13}
\date{22 april 2008}
\usepackage{../sogs08-color}

\newcommand*\emner{%
  \bigskip
  \noindent \rule{5em}{.7pt}

  \noindent Emner som ber�res af denne eksamensopgave}

\begin{document}

\begin{frame}[plain]
  \titlepage
\end{frame}

\part{Forord}

\section{Eksamen}

\begin{frame}
  \begin{textblock*}{8em}[1,.5](\textwidth,.45\textheight)
    \includegraphics[width=\linewidth]{jcg}
  \end{textblock*}

  \begin{itemize}
  \item mundtlig
  \item 7-trins-skala
  \item 10 eksamenssp�rgsm�l, \alert{kendt p� forh�nd}. \emph{(Coming
      up)}
  \item 20 minutters forberedelse
  \item 20 minutters eksamen
  \item hj�lpemidler: ingen computer, ingen mobil-\\telefon,
    \alert{ingen slides}
  \item ekstern censor: Jens Chr.\ Godskesen, ITU

    \url{http://www.itu.dk/~jcg/}
  \item syntaks- og semantikopgaven plus 8 andre
  \item de andre: incl.\ \alert{pr�veopgave}
  \item pr�veopgaven d�kker \alert{kun en del af} opgavens pensum
  \item pr�veopgavens besvarelse indg�r \alert{som en del af} en
    samlet pr�sentation
  \end{itemize}
\end{frame}

\section{Eksamenspensum}

\begin{frame}
  \begin{itemize}
  \item \textit{Sipser} kapitel \alert1, side 31 (31) til 82 (82)
  \item \textit{Sipser} kapitel \alert2, side 101 (99) til 108 (106)
    og 111 (109) til 129 (127)
    \begin{itemize}
    \item dvs.\ alt bortset fra Chomsky-normalformen
    \end{itemize}
  \item \textit{H�ttel} kapitel \alert3, side 35 til 49
  \item \textit{H�ttel} kapitel \alert4, side 51 til 66 og 69 til 70
    \begin{itemize}
    \item dvs.\ alt bortset fra s�tning 4.13, lemma 4.14 og tilh�rende
      beviser
    \end{itemize}
  \item \textit{H�ttel} kapitel \alert5, side 73 til 76 og 78 til 87
    \begin{itemize}
    \item dvs.\ alt bortset fra anden halvdel af beviset for s�tning 5.2
    \end{itemize}
  \item \textit{H�ttel} kapitel \alert6, side 89 til 98 og 102 til 103
    \begin{itemize}
    \item dvs.\ alt bortset fra afsnittene 6.6.1 og 6.6.2
    \end{itemize}
  \item \textit{H�ttel} kapitel \alert7, side 105 til 111
    \begin{itemize}
    \item dvs.\ afsnittene 7.1 til 7.4
    \end{itemize}
  \item \textit{H�ttel} kapitel \alert{14}, side 187 til 201
    \begin{itemize}
    \item dvs.\ afsnittene 14.1 til 14.6
    \end{itemize}
  \end{itemize}
\end{frame}

\section[Eksempel]{Eksempel p� pr�veopgave}

\begin{frame}
  \alert{Opgave 1} fra sidste �r:

  \medskip Lad $L$ v�re sproget givet ved det regul�re udtryk $( a^*bb)^*$.
  \begin{enumerate}
  \item Konstru�r en nondeterministisk endelig automat $A$ som
    genkender $L$.
  \item Forenkl evt.\ $A$ s� den kun har tre tilstande, og konvert�r
    $A$ til en deterministisk endelig automat.
  \end{enumerate}

  \emner:
  \begin{itemize}
  \item deterministiske og nondeterministiske endelige automater
  \item regul�re udtryk
  \item konverteringer mellem de tre
  \item regul�re sprog og deres lukningsegenskaber
  \end{itemize}
\end{frame}

\begin{frame}
  \hfill
  {\tiny\alert{Sp�rgsm�l?}}\hfill\mbox{}
\end{frame}

\section{Semantikopgaven}

\begin{frame}[fragile]
  \textbf{Bof}, variant af \textbf{Bip} med \emph{funktioner}
  (`udtryksprocedurer') i stedet for procedurer:
  \begin{itemize}
  \item �n call-by-value-parameter
  \item returnerer en v�rdi
  \item[\IMPL] funktionskald er ikke \emph{kommando}, men
    \alert{aritmetisk udtryk}
  \item eksempel:
\begin{verbatim}
func f (x) is 
  begin
    var i:= 1;
    f := x + i
  end
...
y := f(17) + 4
\end{verbatim}
  \item dvs.\ v�rdien returneres ved at tilskrive den til en
    (`alias')variabel med samme navn som funktionen
  \end{itemize}
\end{frame}

\begin{frame}
  \begin{itemize}
  \item ny syntaktisk kategori:
    \begin{description}
    \item[$f\in \textbf{Fnavne}$] -- funktionsnavne
    \end{description}
    -- vi s�tter $\textbf{Fnavne}= \textbf{Var}$
  \item nye opbygningsregler:
    \begin{alignat*}{2}
      &\textbf{ErkF:}\quad & D_F &\;::=\; \texttt{func $f(x)$ is
        $S$}\,; D_F \mid \epsilon \\
      &\textbf{Kom:} & S &\;::=\; \cdots\; \mid \texttt{begin
        $D_V$ $D_F$ $S$ end} \\
      &\textbf{Aud:} & a &\;::=\; \cdots\; \mid f(a)
    \end{alignat*}
  \end{itemize}
\end{frame}

\end{document}
