\documentclass{beamer}

\newcommand*{\lektion}9
\date{15 marts 2007}
\usepackage{../sogs07-color}

\begin{document}

\begin{frame}[plain]
  \titlepage
\end{frame}

\part{Semantik}

\section{Syntaks vs.\ semantik}

\begin{frame}
  \alert{Syntaks:} L�ren om sprogs \emph{form}
  \begin{itemize}
  \item hvordan \emph{ser\/} et lovligt program \emph{ud\/}?
  \item beskriv byggesten (\emph{alfabet\/}) og hvordan de kan s�ttes
    sammen (\emph{grammatik}, \emph{automat\/} etc.)
  \end{itemize}

  \alert{Semantik:} L�ren om sprogs \emph{betydning}
  \begin{itemize}
  \item hvordan \emph{opf�rer\/} et givet program sig?
  \item beskriv \emph{betydningen\/} af byggesten og hvordan betydningen
    af sammens�tninger af byggesten f�s ud fra de enkelte betydninger
  \end{itemize}
\end{frame}

\section[Tilgange]{Forskellige tilgange til semantik}

\begin{frame}
  \begin{itemize}
  \item \alert{denotationel} semantik
    \begin{itemize}
    \item beskriv et programs betydning som funktion fra \emph{input\/}
      til \emph{output}
    \item \emph{Hvad\/} laver det her program?
    \end{itemize}
  \item \alert{operationel} semantik
    \begin{itemize}
    \item beskriv et programs betydning som \emph{transitionssystem}
    \item \emph{Hvordan\/} udf�res det her program?
    \end{itemize}
  \item \alert{aksiomatisk} semantik
    \begin{itemize}
    \item beskriv et program ved \emph{pr�-\/} og \emph{post-betingelser}
    \item \emph{Hvilke egenskaber\/} har det her program?
    \end{itemize}
  \item (\alert{algebraisk} semantik: variant af aksiomatisk semantik)
  \end{itemize}
\end{frame}

\section{Anvendelser}

\begin{frame}
  \begin{itemize}
  \item \alert{pr�cis beskrivelse} af programmeringssprog
    \begin{itemize}
    \item ``rettesnor'' til implementation
    \end{itemize}
  \item \alert{automatisk generering} af compilere og fortolkere
  \item \alert{automatisk verifikation} af programmer
    \begin{itemize}
    \item det kan v�re \emph{dyrt\/} at finde fejl i et program ved
      aftestning
    \item[\IMPL] heller finde fejl \emph{f�r}
    \end{itemize}
  \end{itemize}
\end{frame}

\part{Operationel semantik}

\section[Bims]{Abstrakt syntaks for \protect\textbf{Bims}}

\begin{frame}[t]
  \begin{textblock*}{\textwidth}[0,1](0pt,\textheight)
    \small
    \begin{pgfpicture}{0pt}{0pt}{\textwidth}{\textheight}
      \pgfsetxvec{\pgfpoint{1em}{0ex}}
      \pgfsetyvec{\pgfpoint{0em}{1em}}
%      \color{yellow}\pgfsetlinewidth{.2pt}\pgfgrid[step=\pgfpoint{2em}{2em}]{\pgfpoint{0pt}{0pt}}{\pgfpoint{\textwidth}{\textheight}}
      \color{blue}
      \pgfsetlinewidth{.5pt}
      \pgfsetendarrow{\pgfarrowto}
      \pgfputat{\pgfxy(2,6)}{\pgfbox[left,bottom]{basiselementer}}
      \pgfxyline(5.5,7)(5.7,8)
      \pgfxyline(6,7)(6.9,8)
      \pgfputat{\pgfxy(2.5,4)}{\pgfbox[left,bottom]{sammensatte elementer}}
      \pgfxyline(9,5)(10,8)
      \pgfxyline(10,5)(14.2,8)
      \pgfxyline(11,5)(18,8)
      \pgfxyline(12,5)(22,8)
      \color{magenta}
      \pgfputat{\pgfxy(14,2)}{\pgfbox[center,bottom]{umiddelbare bestanddele}}
      \pgfxyline(14,3)(13.1,8)
      \pgfxyline(14.5,3)(15.2,8)
    \end{pgfpicture}
  \end{textblock*}

  \begin{description}
  \item[$n \in \mbox{\bf Num}$] -- Numeraler
  \item[$x \in \mbox{\bf Var}$] --  Variable
  \item[$a \in \mbox{\bf Aud}$] --  Aritmetiske udtryk
  \item[$b \in \mbox{\bf Bud}$] --  Boolske udtryk
  \item[$S \in \mbox{\bf Kom}$] --  Kommandoer
  \end{description}
  \begin{align*} 
    S &\;::=\;  x \mathtt{:=} a \mid \mbox{\tt skip} \mid S_{1};S_{2} 
    \mid \mbox{\tt if}\;\; b \;\;  \mbox{\tt then} \;\; S_{1} \;\;
    \mbox{\tt else} \;\; S_{2}
    \\ &\qquad\mid \mbox{\tt while} \;\; b \;\; \mbox{\tt do} \;\; S
    \\ b &\;::=\; a_{1} = a_{2} \mid a_{1}  <  a_{2} \mid \neg b_{1} \mid
    b_{1} \wedge b_{2} \mid \mbox{\tt (}\, b_{1}\, \mbox{\tt )}
    \\ a &\;::=\; n \mid x \mid a_{1} \splus a_{2} \mid a_{1} \smult a_{2}
    \mid a_{1} \ssub a_{2} \mid \mbox{\tt (}\, a_{1}\, \mbox{\tt )}
  \end{align*}
\end{frame}

\section{Transitionssystemer}

\begin{frame}
  \alert{Definition 3.2:} Et \alert{transitionssystem} er en tripel $(
  \Gamma, \mathord\to, T)$, hvor delene er
  \begin{enumerate}
  \item $\Gamma$ : en m�ngde af \alert{konfigurationer} (eller
    \alert{tilstande})
  \item $\mathord\to\subseteq \Gamma\times \Gamma$ :
    \alert{transitions-relationen}
  \item $T\subseteq \Gamma$ : m�ngden af \alert{slut-konfigurationer}
  \end{enumerate}
  -- en \emph{orienteret graf}

  \medskip Det foruds�ttes desuden at slutkonfigurationerne er
  \alert{terminale}, dvs.\ ikke har nogen udg�ende transitioner: \\
  for ethvert $\gamma\in T$ findes der ingen $\gamma'\in \Gamma$ med
  $\gamma\to \gamma'$.

  \bigskip Operationel semantik $=$ at overs�tte et \emph{program\/} til
  et \emph{transitionssystem\/}:
  \begin{itemize}
  \item konfigurationer $=$ programtilstande
  \item transitioner $=$ programskridt
  \end{itemize}
\end{frame}

\begin{frame}[t]
  \alert{Eksempel:} En operationel semantik for \emph{endelige automater\/}:

  \medskip Givet en NFA $M= ( Q, \Sigma, \delta, q_0, F)$:
  \begin{itemize}
  \item konfigurationer: tilstand i $Q$ plus tilbagev�rende del af
    inputstrengen
    
    dvs.\ \alert{$\Gamma= Q\times \Sigma^*$} \quad\blue{\emph{(uendeligt
        mange konfigurationer!)}}
  \item slutkonfigurationer: sluttilstand i $F$ plus tom streng

    dvs.\ \alert{$T=\{( q, \epsilon)\mid q\in F\}$}
  \item transitioner: at l�se et tegn (eller $\epsilon$) og g� i en anden
    tilstand

    dvs.\ \alert{$( q, aw)\to( q', w)$} hver gang $q'\in \delta( q, a)$,
    og for alle $w\in \Sigma^*$
  \end{itemize}

  \bigskip $M$ accepterer en streng $w$ hvis og kun hvis der findes
  $\gamma\in T$ s�ledes at \alert{$( q_0, w)\overset*\to \gamma$}.
\end{frame}

\begin{frame}
  \alert{Eksempel:} En operationel semantik for \emph{kontekstfrie
    grammatikker\/}:

  \medskip Givet en CFG $G=( V, \Sigma, R, S)$:
  \begin{itemize}
  \item konfigurationer: strenge af variable og terminaler: \\
    \alert{$\Gamma=( V\cup \Sigma)^*$}
  \item slutkonfigurationer: strenge af terminaler: \\
    \alert{$T= \Sigma^*$}
  \item transitioner: derivationsskridt! \\
    \alert{$uAv\To uwv$} hvis $A\to w$ er i $R$
  \end{itemize}

  \bigskip $G$ genererer en streng $w\in T$ hvis og kun hvis
  $S\overset*\To w$.
\end{frame}

\begin{frame}
  \alert{Definition 3.11:} Lad $( \Gamma, \mathord{\Tto{}}, T)$ v�re et
  transitionssystem. \alert{Transitionsaflukningen} i $k$ skridt
  \;\alert{$\Tto k$}\; er defineret induktivt ved
  \begin{align*}
    \gamma \Tto0 \gamma &\quad\text{for alle }\gamma \\
    \gamma \Tto{n+1} \gamma' &\quad\text{hvis der findes } \gamma''\text{ for
      hvilket } \gamma \Tto{} \gamma'' \Tto{n} \gamma'
  \end{align*}
  Vi skriver $\gamma \Tto* \gamma'$ hvis der findes et $k$ s� $\gamma
  \Tto k \gamma'$.

  \bigskip -- dvs.\ $\gamma\Tto k \gamma'$ hvis der findes en
  \emph{transitionsf�lge}
  \begin{equation*}
    \gamma\Tto{} \gamma_1\Tto{} \gamma_2\Tto{}\dots\Tto{}
    \gamma_{k-1}\Tto{} \gamma'
  \end{equation*}

  \medskip -- vi har allerede brugt aflukningen $\Tto*$ adskillige gange!
\end{frame}

\section[Aud:\ big-step]{Big-step-semantik for aritmetiske udtryk (uden
  variable)}

\begin{frame}
  \alert{Aritmetiske udtryk} uden variable:
  \begin{equation*}
    \mbox{\bf Aud:}\qquad a \;::=\; n \mid a_{1} \splus a_{2} \mid a_{1}
    \smult a_{2} \mid a_{1} \ssub a_{2} \mid \mbox{\tt (}\, a_{1}\,
    \mbox{\tt )}
  \end{equation*}
  hvor $n$ er et \alert{numeral} (talord) (en \emph{streng\/}!),
  \emph{ikke et tal}
  \begin{itemize}
  \item numeraler skrives $\ul{42}$, tal skrives $42$
  \item \emph{v�rdien\/} af $\ul{42}$ er $42$
  \item vi har en \emph{semantisk funktion} $\mathcal N: \mbox{\bf Num}\to
    \Int$ som giver v�rdien af en numeral
  \end{itemize}

  \medskip \alert{Big-step}-semantik:\ udtryk evalueres \alert{i �t hug}
  \begin{itemize}
  \item transitioner fra \emph{udtryk\/} til \emph{v�rdier}
  \item f.x.\ en transition $\slp \ul2\splus
    \ul4\srp\smult\slp \ul6\splus \ul1\srp\to 42$
  \end{itemize}
\end{frame}

\begin{frame}[t]
  \begin{textblock*}{\textwidth}[0,1](0pt,\textheight)
    \small \only<2>{%
      \begin{pgfpicture}{0pt}{0pt}{\textwidth}{\textheight}
        \pgfsetxvec{\pgfpoint{2em}{0ex}}
        \pgfsetyvec{\pgfpoint{0em}{2em}}
%        \color{yellow}\pgfsetlinewidth{.2pt}\pgfgrid[step=\pgfpoint{2em}{2em}]{\pgfpoint{0pt}{0pt}}{\pgfpoint{\textwidth}{\textheight}}
        \color{blue}
        \pgfsetlinewidth{.5pt}
        \pgfsetendarrow{\pgfarrowto}
        \pgfputat{\pgfxy(.5,8)}{\pgfbox[left,base]{transitionsregel}}
        \pgfxycurve(2.6,8.4)(2.8,9)(2.9,9.3)(3.3,9.4)
        \pgfputat{\pgfxy(5.55,11)}{\pgfbox[center,base]{pr�mis}}
        \pgfxyline(5.2,10.8)(4.6,10.2)
        \pgfxyline(5.9,10.8)(6.5,10.2)
        \pgfputat{\pgfxy(5,7.5)}{\pgfbox[left,base]{konklusion}}
        \pgfxyline(5.9,8)(5.6,8.6)
        \pgfputat{\pgfxy(10,7.5)}{\pgfbox[left,base]{sidebetingelse}}
        \pgfxyline(11.3,7.9)(10.85,9.1)
        \pgfputat{\pgfxy(3,5)}{\pgfbox[left,base]{aksiom
            \emph{(transitionsregel uden pr�mis)}}}
        \pgfxycurve(3.5,4.9)(3.7,4.3)(3.8,4.3)(4.2,4.2)
      \end{pgfpicture}}
  \end{textblock*}

  \vspace*{-.5em}
  \begin{tabular}{lc}
    \uncover<1,3->{%
      \runa{plus}{bss} & \newcondinfrule{a_{1} \to v_{1} \;\;\; a_{2}
        \to v_{2}}{a_{1} \splus a_{2} \to v}{\mbox{hvor}\; v = v_{1} +
        v_{2}} \\}
    \runa{minus}{bss} & \newcondinfrule{a_{1} \to v_{1} \;\;\; a_{2}
      \to v_{2}}{a_{1} \ssub a_{2} \to v}{\mbox{hvor}\; v = v_{1} - v_{2}} \\
    \uncover<1,3->{%
      \runa{mult}{bss} & \newcondinfrule{a_{1} \to v_{1} \;\;\; a_{2}
        \to v_{2}}{a_{1} \mbox{\tt *} a_{2} \to v}{\mbox{hvor}\; v = v_{1} \cdot v_{2}} \\
      \runa{parent}{bss} & \infrule{a_{1} \to v_{1}}{\mbox{\tt (}
        a_{1} \mbox{\tt )} \to v_{1}} \\}
    \runa{num}{bss} & $n \to v \;\;\;\; \mbox{hvis} \;\;\;\;
    {\cal N}\sem{n} = v$
  \end{tabular}

  \bigskip\medskip \uncover<3->{%
    Transitionssystemet $( \Gamma, \mathord\to, T)$:
    \begin{itemize}
    \item $\Gamma= \mbox{\bf Aud}\cup \Int$, $T= \Int$
    \item $\mathord\to$ best�r af \alert{pr�cis de transitioner som kan
        udledes af aksiomerne ved brug af et endeligt antal
        transitionsregler}
    \end{itemize}}
\end{frame}

\section{Derivationstr�er}

\begin{frame}[t]
  At konstruere et \alert{derivationstr�} for udtrykket $\slp
  \ul2\splus \ul4\srp\smult\slp \ul6\splus
  \ul1\srp$:

  \begin{pgfpicture}{0pt}{0pt}{\textwidth}{.5\textheight}
        \pgfsetxvec{\pgfpoint{2em}{0ex}}
        \pgfsetyvec{\pgfpoint{0em}{2em}}
%        \color{yellow}\pgfsetlinewidth{.2pt}\pgfgrid[step=\pgfpoint{2em}{2em}]{\pgfpoint{0pt}{0pt}}{\pgfpoint{\textwidth}{.5\textheight}}
        \color{black}\pgfsetlinewidth{.4pt}
        \pgfputat{\pgfxy(7,.5)}{\pgfbox[center,center]{$\slp \ul2\splus
            \ul4\srp\smult\slp \ul6\splus \ul1\srp\to
            \alt<4->{\alert<4>{42}}{\blue{\text?\;\;}}$}}
        \uncover<2->{%
          \pgfxyline(3.2,1.3)(10.9,1.3)
          \pgfputat{\pgfxy(4.5,2)}{\pgfbox[center,center]{$\slp \ul2\splus
              \ul4\srp\to \alt<4->{\alert<4>6}{\blue{\text?}}$}}
          \pgfputat{\pgfxy(9.5,2)}{\pgfbox[center,center]{$\slp
              \ul6\splus \ul1\srp\to \alt<4->{\alert<4>7}{\blue{\text?}}$}}
        }%
        \uncover<3->{%
          \pgfxyline(3.3,2.8)(5.8,2.8)
          \pgfxyline(8.3,2.8)(10.8,2.8)
          \pgfputat{\pgfxy(4.6,3.5)}{\pgfbox[center,center]{$\ul2\splus
              \ul4\to \alt<4->{\alert<4>6}{\blue{\text?}}$}}
          \pgfputat{\pgfxy(9.6,3.5)}{\pgfbox[center,center]{$\ul6\splus
              \ul1\to \alt<4->{\alert<4>7}{\blue{\text?}}$}}
        }%
        \uncover<4->{%
          \pgfxyline(2.8,4.3)(6.3,4.3)
          \pgfxyline(7.8,4.3)(11.3,4.3)
          \pgfputat{\pgfxy(3.5,5)}{\pgfbox[center,center]{$\ul2\to \alert<4>2$}}
          \pgfputat{\pgfxy(5.7,5)}{\pgfbox[center,center]{$\ul4\to \alert<4>4$}}
          \pgfputat{\pgfxy(8.5,5)}{\pgfbox[center,center]{$\ul6\to \alert<4>6$}}
          \pgfputat{\pgfxy(10.7,5)}{\pgfbox[center,center]{$\ul1\to \alert<4>1$}}
        }%
        \uncover<5>{%
          \color{red}\pgfsetlinewidth{.7pt}\pgfsetendarrow{\pgfarrowto}
          \pgfxyline(7,.9)(4.6,1.6)
          \pgfxyline(7,.9)(9.6,1.6)
          \pgfxyline(4.6,2.4)(4.6,3.1)
          \pgfxyline(9.6,2.4)(9.6,3.1)
          \pgfxyline(4.6,3.9)(3.5,4.6)
          \pgfxyline(4.6,3.9)(5.7,4.6)
          \pgfxyline(9.6,3.9)(8.5,4.6)
          \pgfxyline(9.6,3.9)(10.7,4.6)
        }
  \end{pgfpicture}

  \uncover<5>{%
    \bigskip derivationstr�er:
    \begin{itemize}
    \item aksiomer i bladene
    \item knude $k$ har s�nner $p_1, p_2,\dots, p_n$ hvis og kun hvis der
      er en transitionsregel

      \vspace{-3ex}
      \begin{equation*}
        \infrule{p_{1}, p_2,\dots, p_{n}}{k}
      \end{equation*}
    \end{itemize}}
\end{frame}

\section[Aud:\ small-step]{Small-step-semantik for aritmetiske udtryk (uden
  variable)}

\begin{frame}
  \alert{Small-step}-semantik:\ udtryk evalueres \alert{et skridt ad gangen}
  \begin{itemize}
  \item transitioner fra \emph{udtryk\/} til \emph{udtryk\/} og fra
    \emph{udtryk\/} til \emph{v�rdier}
  \item f.x.

    \vspace{-4ex}
    \begin{align*}
      \slp \ul2\splus \ul4\srp\smult\slp \ul6\splus \ul1\srp
       &\To \slp 2\splus \ul4\srp\smult\slp \ul6\splus \ul1\srp
      \\ &\To \slp 2\splus 4\srp\smult\slp \ul6\splus \ul1\srp
      \\ &\To \slp 6\srp\smult\slp \ul6\splus \ul1\srp
    \end{align*}
  \item transitionssystem $(\Gamma, \mathord\To, T)$:
    \begin{itemize}
    \item $\Gamma= \mbox{\bf Aud}'\cup \Int$, $T= \Int$
    \item $\mathord\To$ defineret ved transitionsregler
      \blue{\emph{(coming up!)}}
    \end{itemize}
  \end{itemize}

  \bigskip Aritmetiske udtryk uden variable, men \emph{med v�rdier\/}:
  \begin{equation*}
    \mbox{\bf Aud}':\qquad a \;::=\; n \mid v\mid a_{1} \splus a_{2} \mid
    a_{1} \smult a_{2} \mid a_{1} \ssub a_{2} \mid \mbox{\tt (}\, a_{1}\,
    \mbox{\tt )}
  \end{equation*}
  hvor $n\in \mbox{\bf Num}$ er et numeral og $v\in \Int$ en v�rdi
\end{frame}

\begin{frame}
\begin{tabular}{lc}
\runa{plus-1}{sss} & \infrule{a_{1} \To a'_{1}}{a_{1} \splus a_{2} \To
  a'_{1} \splus a_{2}} \\[-2ex]
\runa{plus-2}{sss} & \infrule{a_{2} \To a'_{2}}{a_{1}
  \splus a_{2} \To a_{1} \splus a'_{2} } \\
\runa{plus-3}{sss} & $v_{1} \splus v_{2} \To v \;\;\; \mbox{hvor}\;\;
v = v_{1}+v_{2}$ \\
\runa{mult-1}{sss} & \infrule{a_{1} \To a'_{1}}{a_{1} \smult a_{2} \To
  a'_{1} \smult a_{2}} \\[-2ex]
\runa{mult-2}{sss} & \infrule{a_{2} \To a'_{2}}{a_{1}
  \smult a_{2} \To a_{1} \smult a'_{2} } \\
\runa{mult-3}{sss} & $v_{1} \smult v_{2} \To v \;\;\; \mbox{hvor}\;\;
v = v_{1} \cdot v_{2}$
\end{tabular}
\end{frame}

\begin{frame}
  \begin{tabular}{lc}
\runa{sub-1}{sss} & \infrule{a_{1} \To a'_{1}}{a_{1} \ssub a_{2} \To
  a'_{1} \ssub a_{2}} \\[-2ex]
\runa{sub-2}{sss} & \infrule{a_{2} \To a'_{2}}{a_{1}
  \ssub a_{2} \To a_{1} \ssub a'_{2} } \\
\runa{sub-3}{sss} & $v_{1} \ssub v_{2} \To v \;\;\; \mbox{hvor}\;\;
v = v_{1}-v_{2}$ \\[1ex]
\runa{parent-1}{sss} & \infrule{a_{1} \To a'_{1}}{\mbox{\tt (}
  a_{1} \mbox{\tt )} \To \mbox{\tt (} a'_{1}\mbox{\tt )}} \\[1ex]
\runa{parent-2}{sss} & $\mbox{(} v \mbox{)} \To v$ \\[2ex]
\runa{num}{sss} & $n \To v \;\;\;\; \mbox{hvis} \;\;\;\;
{\cal N}\sem{n} = v$
  \end{tabular}
\end{frame}

\section{Egenskaber}

\begin{frame}
  \alert{S�tning:} Vores big-step- og small-step-semantikker for
  \textbf{Aud} er �kvivalente: Givet $a\in \textbf{Aud}$ og $v\in \Int$,
  da har vi \alert{$a\to v$ hvis og kun hvis $a\overset*\To v$}.
  \hfill\emph{\blue{(Bevis n�ste gang)}}

  \bigskip \alert{Definition:} En operationel semantik givet ved et
  transitionssystem $( \Gamma, \mathord\to, T)$ kaldes
  \alert{deterministisk} hvis $\gamma\to \gamma_1$ og $\gamma\to \gamma_2$
  medf�rer $\gamma_1= \gamma_2$ for alle $\gamma\in \Gamma$ og $\gamma_1,
  \gamma_2\in T$ (\alert!).

  Semantikken kaldes \alert{deterministisk p� lang sigt} hvis
  $\gamma\overset*\to \gamma_1$ og $\gamma\overset*\to \gamma_2$ medf�rer
  $\gamma_1= \gamma_2$ for alle $\gamma\in \Gamma$ og $\gamma_1,
  \gamma_2\in T$.

  \bigskip \alert{S�tning 3.13 / 3.15 :} Vores big-step-semantik for
  \textbf{Aud} er \emph{deterministisk}. Vores small-step-semantik for
  \textbf{Aud} er \emph{deterministisk p� lang
    sigt}. \hfill\emph{\blue{(Bevises senere)}}

  \bigskip \alert{Opgave $\pi$:} Vores small-step-semantik for
  \textbf{Aud} er \emph{ikke deterministisk}. Lav den om s� den er!
\end{frame}

\section[Bud:\ big-step]{Big-step-semantik for boolske udtryk (uden
  variable)}

\begin{frame}[t]
  \alert{Boolske udtryk} uden variable:
  \begin{equation*}
    \qquad\textbf{Bud:}\qquad b \;::=\; a_{1} = a_{2} \mid a_{1}  <  a_{2} \mid
    \neg b_{1} \mid b_{1} \wedge b_{2} \mid \mbox{\tt (}\, b_{1}\,
    \mbox{\tt )}
  \end{equation*}

%  \vspace{-2ex}
  \begin{itemize}
  \item transitionssystem $( \textbf{Bud}\cup\{ \sand, \falsk\},
    \mathord\to_b, \{ \sand, \falsk\})$
  \item $\sand=$ sandt, $\falsk=$ falsk
  \item $\mathord\to_a$ er transitioner fra \textbf{Aud}-transitionssystemet
  \end{itemize}
  
  \rule{10em}{.2pt}

  \begin{tabular}{ll}
    \runa{ligmed-1}{bss} & \newcondinfrule{ a_{1} \to_a v_{1} \;\;\;
      a_{2} \to_a v_{2}}{ a_{1} = a_{2} \to_b \sand}{\mbox{hvis}\; v_{1} =
      v _{2}} \\[-1.2ex] 
    \runa{ligmed-2}{bss} & \newcondinfrule{ a_{1} \to_a v_{1} \;\;\;
      a_{2} \to_a v_{2}}{ a_{1} = a_{2} \to_b \falsk}{\mbox{hvis}\; v_{1}
      \neq v _{2}} \\[-1.2ex] 
    \runa{st�rreend-1}{bss} & \newcondinfrule{ a_{1} \to_a v_{1} \;\;\;
      a_{2} \to_a v_{2}}{ a_{1}  <  a_{2} \to_b \sand}{\mbox{hvis}\; v_{1}
      <  v _{2}} \\[-1.2ex] 
    \runa{st�rreend-2}{bss} & \newcondinfrule{ a_{1} \to_a v_{1} \;\;\;
      a_{2} \to_a v_{2}}{ a_{1}  <  a_{2} \to_b \falsk}{\mbox{hvis}\;
      v_{1} \nless  v _{2}} 
  \end{tabular}
\end{frame}

\begin{frame}
  \begin{tabular}{ll}
    \runa{ikke-1}{bss} & \infrule{ b \to_b \sand}{ \neg b \to_b \falsk} \\ 
    \runa{ikke-2}{bss} & \infrule{ b \to_b \falsk}{ \neg b \to_b \sand} \\ 
    \runa{parent-b}{bss} & \infrule{ b_{1} \to_b v}{ \mbox{\tt (} b_{1}
      \mbox{\tt )} \to_b v} \\ 
    \runa{og-1}{bss} & \infrule{ b_{1} \to_b \sand \;\;\; b_{2} \to_b
      \sand}{ b_{1} \wedge b_{2} \to_b \sand} \\ 
    \runa{og-2}{bss} & \infrule{ b_1 \to_b \falsk}{b_{1} \wedge
      b_{2} \to_b \falsk} \\
    \runa{og-3}{bss} & \infrule{ b_2 \to_b \falsk}{b_{1} \wedge
      b_{2} \to_b \falsk}
  \end{tabular}
\end{frame}

\end{document}
