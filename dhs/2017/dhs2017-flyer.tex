\documentclass[a4paper, 12pt]{article}

\usepackage[hscale=.9, vscale=.86]{geometry}

\usepackage{enumitem}

\parindent 0pt

\hyphenation{distri-buted believe original other-wise}

\pagestyle{empty}

\begin{document}

\begin{center}
  {\Large International Workshop on}

  \medskip

  {\Huge Methods and Tools for Distributed Hybrid Systems}

  \bigskip

  {\Large Aalborg, Denmark, 25-26 August 2017}

  \smallskip

  {\Large Associated with MFCS 2017}
\end{center}

\begin{minipage}[t]{.45\linewidth}
  The purpose of DHS is to connect researchers working in
  \emph{real-time} systems, \emph{hybrid} systems, \emph{control}
  theory, \emph{distributed} computing, and \emph{concurrency}, in
  order to advance the subject of \textbf{distributed hybrid systems}.

  \medskip

  Distributed hybrid systems, or distributed \emph{cyber-physical}
  systems, are abundant. Many of them are safety-critical, but
  ensuring their correct functioning is very difficult. We believe
  that new techniques are needed for the analysis and validation of
  DHS. More precisely, we believe that convergence and interaction of
  methods and tools from different areas of \emph{computer science},
  \emph{engineering}, and \emph{mathematics} is needed in order to
  advance the subject.

  \medskip

  This first edition of the DHS workshop aims at gathering researchers
  which work in the above areas in order to facilitate collaboration
  and discuss how the subject may advance.

  \bigskip

  {\large \bf Call for Contributions}

  \smallskip

  We are calling for contributions presenting original, unfinished,
  already published, or otherwise interesting work which can highlight
  how the above research topics may interact in order to advance the
  subject of distributed hybrid systems. The proceedings of DHS will
  be published as an issue of EPTCS.
\end{minipage}
\hfill
\begin{minipage}[t]{.45\linewidth}
  \hfill {\large \bf Invited Speakers}

  \smallskip

  \textbf{Martin Fr{\"a}nzle}

  \hfill Carl von Ossietzky Universit{\"a}t Oldenburg

  \hfill Germany

  \textbf{Kim G. Larsen}

  \hfill Aalborg Universitet, Denmark

  \smallskip

  % \textbf{Sergio Rajsbaum}

  % \hfill Universidad Nacional Autonoma de Mexico

  % \smallskip

  \textbf{Martin Raussen}

  \hfill Aalborg Universitet, Denmark

  \smallskip

  \textbf{Rafael Wisniewski}

  \hfill Aalborg Universitet, Denmark

  \bigskip\medskip

  \hfill {\large \bf Submissions}

  \medskip

  Submissions to DHS can be of two kinds:
  \begin{itemize}[leftmargin=*, nosep]
  \item Regular papers, containing original contributions presenting
    hitherto unpublished work. If accepted, these papers will be
    published in the EPTCS workshop proceedings.

    \smallskip

  \item Extended abstracts of work-in-progress, of work already
    published or submitted elsewhere, position papers, or otherwise
    interesting work. Extended abstracts will not be published in the
    workshop proceedings.
  \end{itemize}

  \smallskip

  Both types of submissions will be carefully evaluated by the program
  committee. The (soft) page limit for both types of submissions is 20.
\end{minipage}

\bigskip\medskip

\begin{minipage}[t]{.3\linewidth}
  {\large \bf Program Committee}

  \smallskip

  Christel Baier

  Alexandre Chapoutot

  Uli Fahrenberg (chair)

  Lisbeth Fajstrup

  Sebastian Gerwinn

  {\'E}ric Goubault

  Christian Johansen

  Fr{\'e}d{\'e}ric Mallet

  Mohammad Reza Mousavi

  Ulrik Nyman

  Adina Panchea

  Karin Quaas

  Christoffer Sloth
\end{minipage}
\hfill
\begin{minipage}[t]{.6\linewidth}
  {\large \bf Sponsors}

  \medskip

  \textbf{CISS} \hfill Center for Embedded Software Systems

  \hfill Aalborg, Denmark

  \smallskip

  \textbf{Chaire ISC} \hfill {\'E}cole polytechnique -- Thales -- FX
  -- DGA 

  \hfill Dassault Aviation -- DCNS Research -- ENSTA ParisTech

  \hfill T{\'e}l{\'e}com ParisTech -- Fondation ParisTech -- FDO ENSTA

  \hfill Paris, France

  \medskip

  {\large \bf Deadline}

  \smallskip

  Paper submission: 20 June 2017

  Notification: 15 July 2017

  \vspace{4ex}

  \hfill {\Huge http:$/\!/$dhs.gforge.inria.fr/}

\end{minipage}

\end{document}
