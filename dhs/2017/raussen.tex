\documentclass[12pt]{article}

\usepackage{a4wide,amsmath,palatino,mathpazo}

\begin{document}
\subsection*{Martin Raussen: Topological models for spaces of executions in HDA}

Higher dimensional automata (HDA) are powerful models for concurrency
in terms of expressiveness. They can be described using cubical
complexes which makes the topic amenable to a
combinatorial/topological analysis. An execution corresponds to a
directed path (d-path) in such a (time-flow directed) state space, and
a d-homotopy (preserving the directions) of d-paths has equivalent
computations as a result. This is why we started investigating the
\emph{space} of all executions (d-paths between given end points) in
an HDA from a topological perspective.  The determination of path
components is particularly important for applications.

Getting to grips with the effects of the non-reversible time-flow is
essential, and one needs to ``twist'' methods from ordinary algebraic
topology in order to make them applicable.  I will discuss particular
directed spaces arising from Higher Dimensional Automata (HDA). There
are various methods identifying the homotopy type of the space of
executions between two states in such an automaton with some finite
complex: in simple cases as prodsimplicial complex -- with products of
simplices as building blocks -- or as a configuration space living in
a product of simplices. In several interesting cases, it is possible
to calculate homology groups and other topological invariants of
execution spaces. We sketch a method recently devised by
Ziemia\'{n}ski identifying -- for a general HDA -- a space of directed
paths with a prodpermutahedral complex arising by glueing various
permutahedra along their boundaries.

Joint work with L.\ Fajstrup (Aalborg), E.\ Goubault, E.\ Haucourt,
S.\ Mimram (\'{E}c.\ Polytechnique, Paris), R.\ Meshulam (Haifa) and
K.\ Ziemia\'{n}ski (Warsaw)



\end{document}
